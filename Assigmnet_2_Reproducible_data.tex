\documentclass[]{article}
\usepackage{lmodern}
\usepackage{amssymb,amsmath}
\usepackage{ifxetex,ifluatex}
\usepackage{fixltx2e} % provides \textsubscript
\ifnum 0\ifxetex 1\fi\ifluatex 1\fi=0 % if pdftex
  \usepackage[T1]{fontenc}
  \usepackage[utf8]{inputenc}
\else % if luatex or xelatex
  \ifxetex
    \usepackage{mathspec}
  \else
    \usepackage{fontspec}
  \fi
  \defaultfontfeatures{Ligatures=TeX,Scale=MatchLowercase}
\fi
% use upquote if available, for straight quotes in verbatim environments
\IfFileExists{upquote.sty}{\usepackage{upquote}}{}
% use microtype if available
\IfFileExists{microtype.sty}{%
\usepackage{microtype}
\UseMicrotypeSet[protrusion]{basicmath} % disable protrusion for tt fonts
}{}
\usepackage[margin=1in]{geometry}
\usepackage{hyperref}
\hypersetup{unicode=true,
            pdfborder={0 0 0},
            breaklinks=true}
\urlstyle{same}  % don't use monospace font for urls
\usepackage{color}
\usepackage{fancyvrb}
\newcommand{\VerbBar}{|}
\newcommand{\VERB}{\Verb[commandchars=\\\{\}]}
\DefineVerbatimEnvironment{Highlighting}{Verbatim}{commandchars=\\\{\}}
% Add ',fontsize=\small' for more characters per line
\usepackage{framed}
\definecolor{shadecolor}{RGB}{248,248,248}
\newenvironment{Shaded}{\begin{snugshade}}{\end{snugshade}}
\newcommand{\KeywordTok}[1]{\textcolor[rgb]{0.13,0.29,0.53}{\textbf{#1}}}
\newcommand{\DataTypeTok}[1]{\textcolor[rgb]{0.13,0.29,0.53}{#1}}
\newcommand{\DecValTok}[1]{\textcolor[rgb]{0.00,0.00,0.81}{#1}}
\newcommand{\BaseNTok}[1]{\textcolor[rgb]{0.00,0.00,0.81}{#1}}
\newcommand{\FloatTok}[1]{\textcolor[rgb]{0.00,0.00,0.81}{#1}}
\newcommand{\ConstantTok}[1]{\textcolor[rgb]{0.00,0.00,0.00}{#1}}
\newcommand{\CharTok}[1]{\textcolor[rgb]{0.31,0.60,0.02}{#1}}
\newcommand{\SpecialCharTok}[1]{\textcolor[rgb]{0.00,0.00,0.00}{#1}}
\newcommand{\StringTok}[1]{\textcolor[rgb]{0.31,0.60,0.02}{#1}}
\newcommand{\VerbatimStringTok}[1]{\textcolor[rgb]{0.31,0.60,0.02}{#1}}
\newcommand{\SpecialStringTok}[1]{\textcolor[rgb]{0.31,0.60,0.02}{#1}}
\newcommand{\ImportTok}[1]{#1}
\newcommand{\CommentTok}[1]{\textcolor[rgb]{0.56,0.35,0.01}{\textit{#1}}}
\newcommand{\DocumentationTok}[1]{\textcolor[rgb]{0.56,0.35,0.01}{\textbf{\textit{#1}}}}
\newcommand{\AnnotationTok}[1]{\textcolor[rgb]{0.56,0.35,0.01}{\textbf{\textit{#1}}}}
\newcommand{\CommentVarTok}[1]{\textcolor[rgb]{0.56,0.35,0.01}{\textbf{\textit{#1}}}}
\newcommand{\OtherTok}[1]{\textcolor[rgb]{0.56,0.35,0.01}{#1}}
\newcommand{\FunctionTok}[1]{\textcolor[rgb]{0.00,0.00,0.00}{#1}}
\newcommand{\VariableTok}[1]{\textcolor[rgb]{0.00,0.00,0.00}{#1}}
\newcommand{\ControlFlowTok}[1]{\textcolor[rgb]{0.13,0.29,0.53}{\textbf{#1}}}
\newcommand{\OperatorTok}[1]{\textcolor[rgb]{0.81,0.36,0.00}{\textbf{#1}}}
\newcommand{\BuiltInTok}[1]{#1}
\newcommand{\ExtensionTok}[1]{#1}
\newcommand{\PreprocessorTok}[1]{\textcolor[rgb]{0.56,0.35,0.01}{\textit{#1}}}
\newcommand{\AttributeTok}[1]{\textcolor[rgb]{0.77,0.63,0.00}{#1}}
\newcommand{\RegionMarkerTok}[1]{#1}
\newcommand{\InformationTok}[1]{\textcolor[rgb]{0.56,0.35,0.01}{\textbf{\textit{#1}}}}
\newcommand{\WarningTok}[1]{\textcolor[rgb]{0.56,0.35,0.01}{\textbf{\textit{#1}}}}
\newcommand{\AlertTok}[1]{\textcolor[rgb]{0.94,0.16,0.16}{#1}}
\newcommand{\ErrorTok}[1]{\textcolor[rgb]{0.64,0.00,0.00}{\textbf{#1}}}
\newcommand{\NormalTok}[1]{#1}
\usepackage{graphicx,grffile}
\makeatletter
\def\maxwidth{\ifdim\Gin@nat@width>\linewidth\linewidth\else\Gin@nat@width\fi}
\def\maxheight{\ifdim\Gin@nat@height>\textheight\textheight\else\Gin@nat@height\fi}
\makeatother
% Scale images if necessary, so that they will not overflow the page
% margins by default, and it is still possible to overwrite the defaults
% using explicit options in \includegraphics[width, height, ...]{}
\setkeys{Gin}{width=\maxwidth,height=\maxheight,keepaspectratio}
\IfFileExists{parskip.sty}{%
\usepackage{parskip}
}{% else
\setlength{\parindent}{0pt}
\setlength{\parskip}{6pt plus 2pt minus 1pt}
}
\setlength{\emergencystretch}{3em}  % prevent overfull lines
\providecommand{\tightlist}{%
  \setlength{\itemsep}{0pt}\setlength{\parskip}{0pt}}
\setcounter{secnumdepth}{0}
% Redefines (sub)paragraphs to behave more like sections
\ifx\paragraph\undefined\else
\let\oldparagraph\paragraph
\renewcommand{\paragraph}[1]{\oldparagraph{#1}\mbox{}}
\fi
\ifx\subparagraph\undefined\else
\let\oldsubparagraph\subparagraph
\renewcommand{\subparagraph}[1]{\oldsubparagraph{#1}\mbox{}}
\fi

%%% Use protect on footnotes to avoid problems with footnotes in titles
\let\rmarkdownfootnote\footnote%
\def\footnote{\protect\rmarkdownfootnote}

%%% Change title format to be more compact
\usepackage{titling}

% Create subtitle command for use in maketitle
\newcommand{\subtitle}[1]{
  \posttitle{
    \begin{center}\large#1\end{center}
    }
}

\setlength{\droptitle}{-2em}

  \title{}
    \pretitle{\vspace{\droptitle}}
  \posttitle{}
    \author{}
    \preauthor{}\postauthor{}
    \date{}
    \predate{}\postdate{}
  

\begin{document}

\section{\texorpdfstring{title: ``Assignment 2 Reproducible
Data''}{title: Assignment 2 Reproducible Data}}\label{title-assignment-2-reproducible-data}

author: ``Thomas''Trey" Barnes" date: ``July 17, 2019'' output:
html\_document

\begin{Shaded}
\begin{Highlighting}[]
\KeywordTok{download.file}\NormalTok{(}\StringTok{"https://d396qusza40orc.cloudfront.net/repdata%2Fdata%2Factivity.zip"}\NormalTok{, }\DataTypeTok{destfile =} \StringTok{"C:/Users/Trey Barnes/Desktop/RWDir/activity.zip"}\NormalTok{, }\DataTypeTok{mode=}\StringTok{"wb"}\NormalTok{)}
\KeywordTok{unzip}\NormalTok{(}\StringTok{"C:/Users/Trey Barnes/Desktop/RWDir/activity.zip"}\NormalTok{)}
\NormalTok{stepdata <-}\StringTok{ }\KeywordTok{read.csv}\NormalTok{(}\StringTok{"activity.csv"}\NormalTok{, }\DataTypeTok{header =} \OtherTok{TRUE}\NormalTok{)}
\KeywordTok{head}\NormalTok{(stepdata)}
\end{Highlighting}
\end{Shaded}

\begin{verbatim}
##   steps       date interval
## 1    NA 2012-10-01        0
## 2    NA 2012-10-01        5
## 3    NA 2012-10-01       10
## 4    NA 2012-10-01       15
## 5    NA 2012-10-01       20
## 6    NA 2012-10-01       25
\end{verbatim}

\begin{Shaded}
\begin{Highlighting}[]
\KeywordTok{library}\NormalTok{(magrittr)}
\KeywordTok{library}\NormalTok{(dplyr)}
\end{Highlighting}
\end{Shaded}

\begin{verbatim}
## Warning: package 'dplyr' was built under R version 3.5.3
\end{verbatim}

\begin{verbatim}
## 
## Attaching package: 'dplyr'
\end{verbatim}

\begin{verbatim}
## The following objects are masked from 'package:stats':
## 
##     filter, lag
\end{verbatim}

\begin{verbatim}
## The following objects are masked from 'package:base':
## 
##     intersect, setdiff, setequal, union
\end{verbatim}

\subsection{What is mean total number of steps taken per
day?}\label{what-is-mean-total-number-of-steps-taken-per-day}

For this part of the assignment you can ignore the missing values in the
dataset. Calculate the total number of steps taken per day 1 If you do
not understand the difference between a histogram and a barplot,
research the difference between them. 2 Make a histogram of the total
number of steps taken each day 3 Calculate and report the mean and
median of the total number of steps taken per day

\begin{Shaded}
\begin{Highlighting}[]
\NormalTok{knitr}\OperatorTok{::}\NormalTok{opts_chunk}\OperatorTok{$}\KeywordTok{set}\NormalTok{(}\DataTypeTok{echo=} \OtherTok{TRUE}\NormalTok{)}
\NormalTok{databydate <-}\StringTok{ }\NormalTok{stepdata }\OperatorTok\StringTok{ }\KeywordTok{select}\NormalTok{(date, steps) }\OperatorTok\StringTok{ }\KeywordTok{group_by}\NormalTok{(date) }\OperatorTok\StringTok{ }\KeywordTok{summarize}\NormalTok{(}\DataTypeTok{tsteps=} \KeywordTok{sum}\NormalTok{(steps)) }\OperatorTok\KeywordTok{na.omit}\NormalTok{()}
\KeywordTok{hist}\NormalTok{(databydate}\OperatorTok{$}\NormalTok{tsteps, }\DataTypeTok{xlab =} \StringTok{"Total daily Steps"}\NormalTok{,}\DataTypeTok{main=}\StringTok{"Histogram of Total Steps by day"}\NormalTok{, }\DataTypeTok{breaks =} \DecValTok{20}\NormalTok{)}
\end{Highlighting}
\end{Shaded}

\includegraphics{Assigmnet_2_Reproducible_data_files/figure-latex/unnamed-chunk-2-1.pdf}

\begin{Shaded}
\begin{Highlighting}[]
\KeywordTok{mean}\NormalTok{(databydate}\OperatorTok{$}\NormalTok{tsteps)}
\end{Highlighting}
\end{Shaded}

\begin{verbatim}
## [1] 10766.19
\end{verbatim}

\begin{Shaded}
\begin{Highlighting}[]
\KeywordTok{median}\NormalTok{(databydate}\OperatorTok{$}\NormalTok{tsteps)}
\end{Highlighting}
\end{Shaded}

\begin{verbatim}
## [1] 10765
\end{verbatim}

\subsection{Time series plot}\label{time-series-plot}

\begin{Shaded}
\begin{Highlighting}[]
\KeywordTok{library}\NormalTok{(ggplot2)}
\NormalTok{databyinterval <-}\StringTok{ }\NormalTok{stepdata}\OperatorTok\StringTok{ }\KeywordTok{select}\NormalTok{(interval, steps) }\OperatorTok\StringTok{ }\KeywordTok{na.omit}\NormalTok{() }\OperatorTok\StringTok{ }\KeywordTok{group_by}\NormalTok{(interval) }\OperatorTok\StringTok{ }\KeywordTok{summarize}\NormalTok{(}\DataTypeTok{tsteps=} \KeywordTok{mean}\NormalTok{(steps)) }
\KeywordTok{ggplot}\NormalTok{(databyinterval, }\KeywordTok{aes}\NormalTok{(}\DataTypeTok{x=}\NormalTok{interval, }\DataTypeTok{y=}\NormalTok{tsteps))}\OperatorTok{+}\StringTok{ }\KeywordTok{geom_line}\NormalTok{()}
\end{Highlighting}
\end{Shaded}

\includegraphics{Assigmnet_2_Reproducible_data_files/figure-latex/unnamed-chunk-4-1.pdf}

\subsection{The 5-minute interval that, on average, contains the maximum
number of
steps}\label{the-5-minute-interval-that-on-average-contains-the-maximum-number-of-steps}

\begin{Shaded}
\begin{Highlighting}[]
\NormalTok{databyinterval[}\KeywordTok{which}\NormalTok{(databyinterval}\OperatorTok{$}\NormalTok{tsteps}\OperatorTok{==}\StringTok{ }\KeywordTok{max}\NormalTok{(databyinterval}\OperatorTok{$}\NormalTok{tsteps)),]}
\end{Highlighting}
\end{Shaded}

\begin{verbatim}
## # A tibble: 1 x 2
##   interval tsteps
##      <int>  <dbl>
## 1      835   206.
\end{verbatim}

\subsection{Imputing missing values}\label{imputing-missing-values}

\subsection{1. Calculate and report the total number of missing values
in the dataset (i.e.~the total number of rows with
NAs)}\label{calculate-and-report-the-total-number-of-missing-values-in-the-dataset-i.e.the-total-number-of-rows-with-nas}

\subsubsection{generate listing of NA's}\label{generate-listing-of-nas}

\begin{Shaded}
\begin{Highlighting}[]
\NormalTok{missingVals <-}\StringTok{ }\KeywordTok{sum}\NormalTok{(}\KeywordTok{is.na}\NormalTok{(data))}
\end{Highlighting}
\end{Shaded}

\begin{verbatim}
## Warning in is.na(data): is.na() applied to non-(list or vector) of type
## 'closure'
\end{verbatim}

\begin{Shaded}
\begin{Highlighting}[]
\NormalTok{missingVals}
\end{Highlighting}
\end{Shaded}

\begin{verbatim}
## [1] 0
\end{verbatim}

\section{Devise a strategy for filling in all of the missing values in
the dataset. The strategy does not need to be sophisticated. For
example, you could use the mean/median for that day, or the mean for
that 5-minute interval,
etc.}\label{devise-a-strategy-for-filling-in-all-of-the-missing-values-in-the-dataset.-the-strategy-does-not-need-to-be-sophisticated.-for-example-you-could-use-the-meanmedian-for-that-day-or-the-mean-for-that-5-minute-interval-etc.}

\subsubsection{For my strategy I will use the mean for that 5 -minute
interval to replace all the missing values in the dataset. At the end, I
will check if all the NAs have been
replaced}\label{for-my-strategy-i-will-use-the-mean-for-that-5--minute-interval-to-replace-all-the-missing-values-in-the-dataset.-at-the-end-i-will-check-if-all-the-nas-have-been-replaced}

\begin{Shaded}
\begin{Highlighting}[]
\KeywordTok{library}\NormalTok{(magrittr)}
\KeywordTok{library}\NormalTok{(dplyr)}

\NormalTok{replacewithmean <-}\StringTok{ }\ControlFlowTok{function}\NormalTok{(x) }\KeywordTok{replace}\NormalTok{(x, }\KeywordTok{is.na}\NormalTok{(x), }\KeywordTok{mean}\NormalTok{(x, }\DataTypeTok{na.rm =} \OtherTok{TRUE}\NormalTok{))}
\NormalTok{meandata <-}\StringTok{ }\NormalTok{stepdata}\OperatorTok\StringTok{ }\KeywordTok{group_by}\NormalTok{(interval) }\OperatorTok\StringTok{ }\KeywordTok{mutate}\NormalTok{(}\DataTypeTok{steps=} \KeywordTok{replacewithmean}\NormalTok{(steps))}
\KeywordTok{head}\NormalTok{(meandata)}
\end{Highlighting}
\end{Shaded}

\begin{verbatim}
## # A tibble: 6 x 3
## # Groups:   interval [6]
##    steps date       interval
##    <dbl> <fct>         <int>
## 1 1.72   2012-10-01        0
## 2 0.340  2012-10-01        5
## 3 0.132  2012-10-01       10
## 4 0.151  2012-10-01       15
## 5 0.0755 2012-10-01       20
## 6 2.09   2012-10-01       25
\end{verbatim}

\subsection{Make a histogram of the total number of steps taken each day
and Calculate and report the mean and median total number of steps taken
per
day.}\label{make-a-histogram-of-the-total-number-of-steps-taken-each-day-and-calculate-and-report-the-mean-and-median-total-number-of-steps-taken-per-day.}

\begin{Shaded}
\begin{Highlighting}[]
\NormalTok{FullSummedDataByDay <-}\StringTok{ }\KeywordTok{aggregate}\NormalTok{(meandata}\OperatorTok{$}\NormalTok{steps, }\DataTypeTok{by=}\KeywordTok{list}\NormalTok{(meandata}\OperatorTok{$}\NormalTok{date), sum)}

\KeywordTok{names}\NormalTok{(FullSummedDataByDay)[}\DecValTok{1}\NormalTok{] =}\StringTok{"date"}
\KeywordTok{names}\NormalTok{(FullSummedDataByDay)[}\DecValTok{2}\NormalTok{] =}\StringTok{"totalsteps"}
\KeywordTok{head}\NormalTok{(FullSummedDataByDay,}\DecValTok{15}\NormalTok{)}
\end{Highlighting}
\end{Shaded}

\begin{verbatim}
##          date totalsteps
## 1  2012-10-01   10766.19
## 2  2012-10-02     126.00
## 3  2012-10-03   11352.00
## 4  2012-10-04   12116.00
## 5  2012-10-05   13294.00
## 6  2012-10-06   15420.00
## 7  2012-10-07   11015.00
## 8  2012-10-08   10766.19
## 9  2012-10-09   12811.00
## 10 2012-10-10    9900.00
## 11 2012-10-11   10304.00
## 12 2012-10-12   17382.00
## 13 2012-10-13   12426.00
## 14 2012-10-14   15098.00
## 15 2012-10-15   10139.00
\end{verbatim}

\begin{Shaded}
\begin{Highlighting}[]
\KeywordTok{summary}\NormalTok{(FullSummedDataByDay)}
\end{Highlighting}
\end{Shaded}

\begin{verbatim}
##          date      totalsteps   
##  2012-10-01: 1   Min.   :   41  
##  2012-10-02: 1   1st Qu.: 9819  
##  2012-10-03: 1   Median :10766  
##  2012-10-04: 1   Mean   :10766  
##  2012-10-05: 1   3rd Qu.:12811  
##  2012-10-06: 1   Max.   :21194  
##  (Other)   :55
\end{verbatim}

\subsection{Making a histogram}\label{making-a-histogram}

\begin{Shaded}
\begin{Highlighting}[]
\KeywordTok{hist}\NormalTok{(FullSummedDataByDay}\OperatorTok{$}\NormalTok{totalsteps, }\DataTypeTok{xlab =} \StringTok{"Steps"}\NormalTok{, }\DataTypeTok{ylab =} \StringTok{"Frequency"}\NormalTok{, }\DataTypeTok{main =} \StringTok{"Total Daily Steps"}\NormalTok{, }\DataTypeTok{breaks =} \DecValTok{20}\NormalTok{)}
\end{Highlighting}
\end{Shaded}

\includegraphics{Assigmnet_2_Reproducible_data_files/figure-latex/unnamed-chunk-10-1.pdf}

\subsection{Compare the mean and median of Old and New
data}\label{compare-the-mean-and-median-of-old-and-new-data}

\begin{Shaded}
\begin{Highlighting}[]
\NormalTok{oldmean <-}\StringTok{ }\KeywordTok{mean}\NormalTok{(databydate}\OperatorTok{$}\NormalTok{tsteps, }\DataTypeTok{na.rm =} \OtherTok{TRUE}\NormalTok{)}
\NormalTok{newmean <-}\StringTok{ }\KeywordTok{mean}\NormalTok{(FullSummedDataByDay}\OperatorTok{$}\NormalTok{totalsteps)}
\NormalTok{oldmean}
\end{Highlighting}
\end{Shaded}

\begin{verbatim}
## [1] 10766.19
\end{verbatim}

\begin{Shaded}
\begin{Highlighting}[]
\NormalTok{newmean}
\end{Highlighting}
\end{Shaded}

\begin{verbatim}
## [1] 10766.19
\end{verbatim}

\begin{Shaded}
\begin{Highlighting}[]
\NormalTok{oldmedian <-}\StringTok{ }\KeywordTok{median}\NormalTok{(databydate}\OperatorTok{$}\NormalTok{tsteps, }\DataTypeTok{na.rm =} \OtherTok{TRUE}\NormalTok{)}
\NormalTok{newmedian <-}\StringTok{ }\KeywordTok{median}\NormalTok{(FullSummedDataByDay}\OperatorTok{$}\NormalTok{totalsteps)}
\NormalTok{oldmedian}
\end{Highlighting}
\end{Shaded}

\begin{verbatim}
## [1] 10765
\end{verbatim}

\begin{Shaded}
\begin{Highlighting}[]
\NormalTok{newmedian}
\end{Highlighting}
\end{Shaded}

\begin{verbatim}
## [1] 10766.19
\end{verbatim}

Are there differences in activity patterns between weekdays and
weekends?

\begin{Shaded}
\begin{Highlighting}[]
\NormalTok{meandata}\OperatorTok{$}\NormalTok{date <-}\StringTok{ }\KeywordTok{as.Date}\NormalTok{(meandata}\OperatorTok{$}\NormalTok{date)}
\NormalTok{meandata}\OperatorTok{$}\NormalTok{weekday <-}\StringTok{ }\KeywordTok{weekdays}\NormalTok{(meandata}\OperatorTok{$}\NormalTok{date)}
\NormalTok{meandata}\OperatorTok{$}\NormalTok{weekend <-}\StringTok{ }\KeywordTok{ifelse}\NormalTok{(meandata}\OperatorTok{$}\NormalTok{weekday}\OperatorTok{==}\StringTok{"Saturday"} \OperatorTok{|}\StringTok{ }\NormalTok{meandata}\OperatorTok{$}\NormalTok{weekday}\OperatorTok{==}\StringTok{"Sunday"}\NormalTok{, }\StringTok{"Weekend"}\NormalTok{, }\StringTok{"Weekday"}\NormalTok{ )}
\KeywordTok{library}\NormalTok{(ggplot2)}
\NormalTok{meandataweekendweekday <-}\StringTok{ }\KeywordTok{aggregate}\NormalTok{(meandata}\OperatorTok{$}\NormalTok{steps , }\DataTypeTok{by=} \KeywordTok{list}\NormalTok{(meandata}\OperatorTok{$}\NormalTok{weekend, meandata}\OperatorTok{$}\NormalTok{interval), }\KeywordTok{na.omit}\NormalTok{(mean))}
\KeywordTok{names}\NormalTok{(meandataweekendweekday) <-}\StringTok{ }\KeywordTok{c}\NormalTok{(}\StringTok{"weekend"}\NormalTok{, }\StringTok{"interval"}\NormalTok{, }\StringTok{"steps"}\NormalTok{)}

\KeywordTok{ggplot}\NormalTok{(meandataweekendweekday, }\KeywordTok{aes}\NormalTok{(}\DataTypeTok{x=}\NormalTok{interval, }\DataTypeTok{y=}\NormalTok{steps, }\DataTypeTok{color=}\NormalTok{weekend)) }\OperatorTok{+}\StringTok{ }\KeywordTok{geom_line}\NormalTok{()}\OperatorTok{+}
\StringTok{  }\KeywordTok{facet_grid}\NormalTok{(weekend }\OperatorTok{~}\NormalTok{.) }\OperatorTok{+}\StringTok{ }\KeywordTok{xlab}\NormalTok{(}\StringTok{"Interval"}\NormalTok{) }\OperatorTok{+}\StringTok{ }\KeywordTok{ylab}\NormalTok{(}\StringTok{"Mean of Steps"}\NormalTok{) }\OperatorTok{+}
\StringTok{  }\KeywordTok{ggtitle}\NormalTok{(}\StringTok{"Comparison of Average Number of Steps in Each Interval"}\NormalTok{)}
\end{Highlighting}
\end{Shaded}

\includegraphics{Assigmnet_2_Reproducible_data_files/figure-latex/unnamed-chunk-12-1.pdf}


\end{document}
